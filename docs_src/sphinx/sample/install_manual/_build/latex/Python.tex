%% Generated by Sphinx.
\def\sphinxdocclass{jsbook}
\documentclass[letterpaper,10pt,dvipdfmx]{sphinxmanual}
\ifdefined\pdfpxdimen
   \let\sphinxpxdimen\pdfpxdimen\else\newdimen\sphinxpxdimen
\fi \sphinxpxdimen=.75bp\relax



\usepackage{cmap}
\usepackage[T1]{fontenc}
\usepackage{amsmath,amssymb,amstext}

\usepackage{times}

\usepackage[dontkeepoldnames]{sphinx}

\usepackage[dvipdfm]{geometry}

% Include hyperref last.
\usepackage{hyperref}
% Fix anchor placement for figures with captions.
\usepackage{hypcap}% it must be loaded after hyperref.
% Set up styles of URL: it should be placed after hyperref.
\urlstyle{same}

\renewcommand{\figurename}{図}
\renewcommand{\tablename}{表}
\renewcommand{\literalblockname}{リスト}

\renewcommand{\literalblockcontinuedname}{continued from previous page}
\renewcommand{\literalblockcontinuesname}{continues on next page}

\def\pageautorefname{ページ}





\title{Pythonインストール手順書 Documentation}
\date{2018年02月04日}
\release{}
\author{Takeshi KOMIYA}
\newcommand{\sphinxlogo}{\vbox{}}
\renewcommand{\releasename}{リリース}
\makeindex

\begin{document}

\maketitle
\sphinxtableofcontents
\phantomsection\label{\detokenize{index::doc}}



\chapter{文書の目的}
\label{\detokenize{purpose:id1}}\label{\detokenize{purpose:python-2-7}}\label{\detokenize{purpose::doc}}
この手順書ではPython 2.7のインストールを行います。
2013年8月現在、Python 2.7系の最新版は2.7.5です。

この手順書ではCentOS 6.4環境を想定しています。
また以下のパッケージがインストールされているものとします。


\begin{savenotes}\sphinxattablestart
\centering
\sphinxcapstartof{table}
\sphinxcaption{Pythonの想定環境}\label{\detokenize{purpose:id2}}
\sphinxaftercaption
\begin{tabulary}{\linewidth}[t]{|T|T|}
\hline
\sphinxstylethead{\sphinxstyletheadfamily 
ソフトウェア
\unskip}\relax &\sphinxstylethead{\sphinxstyletheadfamily 
バージョン
\unskip}\relax \\
\hline
gcc
&
4.4.7-3.el6
\\
\hline
make
&
3.81-20.el6
\\
\hline
openssl-devel
&
1.0.0-27.el6\_4.2
\\
\hline
bzip2-devel
&
1.0.5-7.el6\_0
\\
\hline
expat-devel
&
2.0.1-11.el6\_2
\\
\hline
db4-devel
&
4.7.25-17.el6
\\
\hline
sqlite-devel
&
3.6.20-1.el6
\\
\hline
ncurses-devel
&
5.7-3.20090208.el6
\\
\hline
readline-devel
&
6.0-4.el6
\\
\hline
gdbm-devel
&
1.8.0-36.el6
\\
\hline
tk-devel
&
8.5.7-5.el6
\\
\hline
\end{tabulary}
\par
\sphinxattableend\end{savenotes}

インストール作業中にroot権限を必要とするため、
あらかじめ作業ユーザがsudoできるよう設定しておく必要があります。


\chapter{具体的な手順}
\label{\detokenize{procedure/index:id1}}\label{\detokenize{procedure/index::doc}}

\section{前提条件の確認}
\label{\detokenize{procedure/precondition:id1}}\label{\detokenize{procedure/precondition::doc}}
各種コマンドを使って前提となる環境がセットアップされているか確認します。
インストール先のマシンにssh等でログインして以下の操作を行なってください。
\begin{description}
\item[{ディストリビューションの確認}] \leavevmode
redhat-releaseファイルの内容を表示し、CentOS 6.4であることを確認します:

\fvset{hllines={, ,}}%
\begin{sphinxVerbatim}[commandchars=\\\{\}]
\PYGZdl{} cat /etc/redhat\PYGZhy{}release
CentOS release 6.4 (Final)
\end{sphinxVerbatim}

\item[{インストールパッケージの確認}] \leavevmode
rpmコマンドを利用して、各インストールパッケージのバージョンを確認します:

\fvset{hllines={, ,}}%
\begin{sphinxVerbatim}[commandchars=\\\{\}]
\PYGZdl{} rpm \PYGZhy{}q gcc make openssl\PYGZhy{}devel bzip2\PYGZhy{}devel expat\PYGZhy{}devel db4\PYGZhy{}devel sqlite\PYGZhy{}devel ncurses\PYGZhy{}devel readline\PYGZhy{}devel gdbm\PYGZhy{}devel tk\PYGZhy{}devel
gcc\PYGZhy{}4.4.7\PYGZhy{}3.el6.x86\PYGZus{}64
make\PYGZhy{}3.81\PYGZhy{}20.el6.x86\PYGZus{}64
openssl\PYGZhy{}devel\PYGZhy{}1.0.0\PYGZhy{}27.el6\PYGZus{}4.2.x86\PYGZus{}64
bzip2\PYGZhy{}devel\PYGZhy{}1.0.5\PYGZhy{}7.el6\PYGZus{}0.x86\PYGZus{}64
expat\PYGZhy{}devel\PYGZhy{}2.0.1\PYGZhy{}11.el6\PYGZus{}2.x86\PYGZus{}64
db4\PYGZhy{}devel\PYGZhy{}4.7.25\PYGZhy{}17.el6.x86\PYGZus{}64
sqlite\PYGZhy{}devel\PYGZhy{}3.6.20\PYGZhy{}1.el6.x86\PYGZus{}64
ncurses\PYGZhy{}devel\PYGZhy{}5.7\PYGZhy{}3.20090208.el6.x86\PYGZus{}64
readline\PYGZhy{}devel\PYGZhy{}6.0\PYGZhy{}4.el6.x86\PYGZus{}64
gdbm\PYGZhy{}devel\PYGZhy{}1.8.0\PYGZhy{}36.el6.x86\PYGZus{}64
tk\PYGZhy{}devel\PYGZhy{}8.5.7\PYGZhy{}5.el6.x86\PYGZus{}64
\end{sphinxVerbatim}

\item[{sudo権限の確認}] \leavevmode
sudoを実行し、root権限を獲得できるか確認します:

\fvset{hllines={, ,}}%
\begin{sphinxVerbatim}[commandchars=\\\{\}]
\PYGZdl{} sudo cat /etc/redhat\PYGZhy{}release
[sudo] password for tkomiya:
CentOS release 6.3 (Final)
\end{sphinxVerbatim}

sudoが設定されていない場合は、次のようなエラーとなります:

\fvset{hllines={, ,}}%
\begin{sphinxVerbatim}[commandchars=\\\{\}]
\PYGZdl{} sudo cat /etc/redhat\PYGZhy{}release
[sudo] password for tkomiya:
tkomiya is not in the sudoers file.  This incident will be reported.
\end{sphinxVerbatim}

\end{description}


\section{インストール手順}
\label{\detokenize{procedure/installation:id1}}\label{\detokenize{procedure/installation::doc}}
インストール先のマシンにssh等でログインして以下の操作を行ってください。
\begin{enumerate}
\item {} 
最初にPythonのアーカイブをpython.orgからダウンロードします:

\fvset{hllines={, ,}}%
\begin{sphinxVerbatim}[commandchars=\\\{\}]
\PYGZdl{} mkdir \PYGZdl{}HOME/src
\PYGZdl{} cd \PYGZdl{}HOME/src
\PYGZdl{} curl \PYGZhy{}L \PYGZhy{}O http://www.python.org/ftp/python/2.7.5/Python\PYGZhy{}2.7.5.tgz
  \PYGZpc{} Total    \PYGZpc{} Received \PYGZpc{} Xferd  Average Speed   Time    Time     Time  Current
                              Dload  Upload   Total   Spent    Left  Speed
100 15.9M  100 15.9M    0     0  4355k      0  0:00:03  0:00:03 \PYGZhy{}\PYGZhy{}:\PYGZhy{}\PYGZhy{}:\PYGZhy{}\PYGZhy{} 4730k
\PYGZdl{} ls
Python\PYGZhy{}2.7.5.tgz
\end{sphinxVerbatim}

curlコマンドを実行するとダウンロードが開始されます。
少し待つとダウンロードが完了し、アーカイブである Python-2.7.5.tgz が保存されています。

\item {} 
ダウンロードしたアーカイブを展開し、Pythonをビルドします:

\fvset{hllines={, ,}}%
\begin{sphinxVerbatim}[commandchars=\\\{\}]
\PYGZdl{} tar xzf Python\PYGZhy{}2.7.5.tgz
\PYGZdl{} cd Python\PYGZhy{}2.7.5
\PYGZdl{} ./configure
\PYGZdl{} make
\end{sphinxVerbatim}

\item {} 
インストール先にファイルを配置します:

\fvset{hllines={, ,}}%
\begin{sphinxVerbatim}[commandchars=\\\{\}]
\PYGZdl{} sudo make install
\end{sphinxVerbatim}

\end{enumerate}


\chapter{確認方法}
\label{\detokenize{verification:id1}}\label{\detokenize{verification::doc}}
hashコマンドを実行した後、pythonのバージョンを確認します:

\fvset{hllines={, ,}}%
\begin{sphinxVerbatim}[commandchars=\\\{\}]
\PYGZdl{} hash \PYGZhy{}r
\PYGZdl{} which python
/usr/local/bin/python
\PYGZdl{} python \PYGZhy{}V
Python 2.7.5
\end{sphinxVerbatim}

バージョンが2.7.5になっていればインストール完了です。



\renewcommand{\indexname}{索引}
\printindex
\end{document}